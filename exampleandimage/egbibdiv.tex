\documentclass{article}
\usepackage{ctex,hyperref}
\usepackage{geometry}
\geometry{paperwidth=18cm,paperheight=20cm,%
left=1cm,right=1cm,top=1cm,bottom=1.5cm}
\usepackage[backend=biber,style=numeric]{biblatex}%gb7714-2015,defernumbers=true
\addbibresource{example.bib}
\renewcommand{\bibfont}{\zihao{6}}
\usepackage{titlesec}
%\titleformat{command}[shape]{format}{label}{sep}{before}[after]
\titleformat{\section}{\centering\bfseries}{第\thesection 节}{1em}{}[]
\titlespacing*{\section}{0pt}{0.0\baselineskip}{0.0\baselineskip}[0pt]
\titleformat{\subsection}{\flushleft\bfseries}{\S\,\thesubsection}{1em}{}[]
\titlespacing*{\subsection}{0pt}{0.0\baselineskip}{0.0\baselineskip}[0pt]
\begin{document}
\footnotesize

 	参考文献测试\cite{Gradshteyn2000--}。
    
    \section{refSegment A}
	\begin{refsegment}
		
		分章节参考文献测试\cite{Chiani2003-840-845}
		\printbibliography[segment=1,heading=subbibliography,title=文献A]
	\end{refsegment}
	
    \section{refSegment B}
	\begin{refsegment}
		参考文献测试\cite{张敏莉2007-500-503}
	\end{refsegment}
	%\printbibliography放在refsegment环境外也是可以的
	\printbibliography[segment=2,heading=subbibliography,title=文献B]
	
    \section{refsection C}
	\begin{refsection}
		参考文献测试\cite{Zhang2007-500-503}
	\end{refsection}

\section{refsection D}
	\begin{refsection}
		分章节参考文献测试\cite{Andersen1995-42-49}
\subsection{refsegment D-1}
    	\begin{refsegment}
			分章节参考文献测试\cite{Simon2004--}。
		\end{refsegment}

\subsection{refsegment D-2}
		\begin{refsegment}
			分章节参考文献测试\cite{Lin2004--}。
		\end{refsegment}
	\end{refsection}

%\printbibliography[section=2,segment=0,heading=subbibliography,title=文献D0]
%\printbibliography[section=2,segment=1,heading=subbibliography,title=文献D1]
%\printbibliography[section=2,segment=2,heading=subbibliography,title=文献D2]
\printbibliography[section=1,heading=subbibliography,title=文献C]
\printbibliography[section=2,heading=subbibliography,title=文献D]

%遍历非refsection内的参考文献
\printbibliography[heading=bibliography,title=文献全局]

%\printbibliography[heading=bibliography,title=文献全局,resetnumbers=true]
%
%\printbibliography[heading=bibliography,title=文献全局,omitnumbers=true]
%
%\begin{refcontext}[sorting=ynt]
%\printbibliography[heading=bibliography,title=文献全局,resetnumbers=true]
%\end{refcontext}

\appendix
\section{Section E}
参考文献测试\cite{Parsons2000--}。
\end{document} 